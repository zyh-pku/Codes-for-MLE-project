%2multibyte Version: 5.50.0.2960 CodePage: 65001
%\usepackage{multirow}
%\usepackage{floatrow}

\documentclass[11pt]{article}%
\usepackage{amsfonts}
\usepackage{amsmath,amssymb,amsthm,enumerate,epsfig,graphicx,natbib}
\usepackage{ifthen,latexsym,syntonly}
\usepackage{natbib}
\usepackage{rotating}
\usepackage{lscape}
\usepackage{amsmath}
\usepackage{amssymb}
\usepackage{graphicx}
\usepackage{color}
\usepackage[bottom]{footmisc}
\usepackage{longtable}%
\setcounter{MaxMatrixCols}{30}
%TCIDATA{OutputFilter=latex2.dll}
%TCIDATA{Version=5.50.0.2960}
%TCIDATA{Codepage=65001}
%TCIDATA{CSTFile=LaTeX Report (perso).cst}
%TCIDATA{LastRevised=Thursday, March 30, 2023 15:37:51}
%TCIDATA{<META NAME="GraphicsSave" CONTENT="32">}
%TCIDATA{<META NAME="SaveForMode" CONTENT="1">}
%TCIDATA{BibliographyScheme=BibTeX}
%TCIDATA{Language=American English}
%BeginMSIPreambleData
\providecommand{\U}[1]{\protect\rule{.1in}{.1in}}
%EndMSIPreambleData
\RequirePackage[colorlinks,citecolor=black,urlcolor=black,linkcolor=black]{hyperref}
\textwidth=6.6in
\textheight=8.9in
\headheight=0.0in
\oddsidemargin=0.0in
\headsep=0.0in
\topmargin=0.0in
\newtheorem{theorem}{Theorem}
\newtheorem{corollary}{Corollary}
\newtheorem{case}{Case}
\newtheorem{lemma}{Lemma}
\newtheorem{proposition}{Proposition}
\newtheorem{assumption}{Assumption}
\theoremstyle{definition}
\newtheorem{definition}{Definition}
\newtheorem{example}{Example}
\newtheorem{remark}{Remark}
\def\baselinestretch{1.3}
\newcommand{\abs}[1]{\lvert#1\rvert}
\newcommand{\norm}[1]{\left\lVert#1\right\rVert}
\DeclareMathOperator*{\adjugate}{adj}
\DeclareMathOperator*{\sign}{sgn}
\allowdisplaybreaks
\begin{document}

\title{Notes on MMLE asymptotics in the SV model}
\date{\today }
\maketitle
\tableofcontents

\section{\textquotedblleft
AA\_our\_OU\_model\_Fisher\_formula\_Final\_correct\_version.nb\textquotedblright%
}

Input: None

Output: Fisher\_OurOUmodel.m

The result in Fisher\_OurOUmodel.m is a 4*4 matrix $I_{\vartheta}^{(1)}$,
where $I_{\vartheta}^{\text{M}}(\Delta)=I_{\vartheta}^{(1)}\Delta+o(\Delta)$
is the long-span Fisher information of MMLE under the OU model in the paper,
say%
\begin{align*}
dX_{t}  &  =\kappa{_{1}({Y}_{t}-X_{t})}dt+\sigma_{1}dW_{1t},\\
dY_{t}  &  =\kappa{_{2}(\mu-{Y}_{t})}dt+\rho\sigma_{2}dW_{1t}+\sqrt{1-\rho
^{2}}\sigma_{2}dW_{2t},
\end{align*}
with%
\[
\vartheta=(\mu,\kappa{_{1},}\kappa_{2},\sigma_{2},\rho)^{\intercal}.
\]


For the \textquotedblleft
AA\_our\_OU\_model\_Fisher\_formula\_Final\_correct\_version\_rho\_is\_0.nb\textquotedblright%
, the model is
\begin{align*}
dX_{t}  &  =\kappa{_{1}({Y}_{t}-X_{t})}dt+\sigma_{1}dW_{1t},\\
dY_{t}  &  =\kappa{_{2}(\mu-{Y}_{t})}dt+\sigma_{2}dW_{2t},
\end{align*}
with%
\[
\vartheta=(\mu,\kappa{_{1},}\kappa_{2},\sigma_{2})^{\intercal}.
\]


In particular, only the first few lines in the codes are designed for the
above special OU model. The codes also adapt to the general OU model as
follows,%
\[
d\left[
\begin{array}
[c]{l}%
{X}_{t}\\
{Y}_{t}%
\end{array}
\right]  =\left[
\begin{array}
[c]{ll}%
{k_{11}} & {k_{12}}\\
{0} & {k_{22}}%
\end{array}
\right]  \left[
\begin{array}
[c]{l}%
{\mu_{1}-X}_{t}\\
{\mu_{2}-Y}_{t}%
\end{array}
\right]  dt+\left[
\begin{array}
[c]{cc}%
{\sigma_{1}} & {0}\\
{\sigma_{2}}\rho & {\sigma_{2}\sqrt{1-\rho^{2}}}%
\end{array}
\right]  d\left[
\begin{array}
[c]{l}%
{W_{1,t}}\\
{W_{2,t}}%
\end{array}
\right]  ,
\]
with ${\sigma_{11}=\sigma_{1},}$ ${\sigma_{21}=\sigma_{2}}\rho,$ ${\sigma
_{22}=\sigma_{2}\sqrt{1-\rho^{2}}}$.

The special OU model we are using is obtained by $\kappa_{11}=-\kappa
_{12}=\kappa_{1}$, $\kappa_{22}=\kappa_{2}$, $\mu_{1}=\mu_{2}=\mu$.

In codes, we directly use the continuous-time SDE to compute the Fisher information.

\subsection{math formulas foundation with codes explanations (no $\sigma_{1}$
v.s. $\sigma_{1,0}$ distinguish)}

We start by writing down the dynamics of all the processes we need.

By Kushner equation, (note that for $v_{t}[X_{[0,t]}^{\vartheta};\vartheta]$,
when we use $(X_{t}^{\vartheta},Y_{t}^{\vartheta})$ to establish its dynamics,
we will get the ODE below without any stochastic parts, so when we replace
$X_{[0,t]}^{\vartheta}$ by $X_{[0,t]}$, the dynamics of $v_{t}[X_{[0,t]}%
;\vartheta]$ is still the ODE below):

filter variance --%
\[
\frac{dv_{t}}{dt}=\sigma_{22}^{2}-\left(  2k_{22}-\frac{2k_{12}}{\sigma_{11}%
}\sigma_{21}\right)  v_{t}-\left(  \frac{k_{12}}{\sigma_{11}}\right)
^{2}v_{t}^{2}%
\]


filter mean -- for $\mu_{t}[X_{[0,t]};\vartheta]$:%
\begin{equation}
d\mu_{t}=P_{1}\left(  \mu_{2}-\mu_{t}\right)  dt+P_{2}\left(  \mu_{1}%
-X_{t}\right)  dt+P_{3}dX_{t},\label{2DOU filter mean SDE}%
\end{equation}
where $P_{1},P_{2}$ and $P_{3}$ are functions of parameters%
\[
P_{1}=k_{22}-k_{12}\left[  -\frac{k_{12}v^{\left(  0\right)  }}{\sigma_{1}%
^{2}}+\frac{\sigma_{21}}{\sigma_{1}}\right]  ,P_{2}=-k_{11}\left[
-\frac{k_{12}v^{\left(  0\right)  }}{\sigma_{1}^{2}}+\frac{\sigma_{21}}%
{\sigma_{1}}\right]  ,P_{3}=\left[  -\frac{k_{12}v^{\left(  0\right)  }%
}{\sigma_{1}^{2}}+\frac{\sigma_{21}}{\sigma_{1}}\right]  ,
\]
where for simplicity of computation we directly plug the stationary value
$v^{\left(  0\right)  }$ of $v_{t}$ into the above dynamics of $\mu_{t}$.

Then we obtain (here $\mu_{t}$ represents $\mu_{t}[X_{[0,t]};\vartheta]$)
\begin{align*}
d\left[  \frac{\partial\mu_{t}}{\partial\vartheta}\right]   &  =\frac
{\partial\left[  P_{1}\left(  \mu_{2}-\mu_{t}\right)  \right]  }%
{\partial\vartheta}dt+\frac{\partial\left[  P_{2}\left(  \mu_{1}-X_{t}\right)
\right]  }{\partial\vartheta}dt+\frac{\partial P_{3}}{\partial\vartheta}%
dX_{t}\\
&  =\frac{\partial\left[  P_{1}\left(  \mu_{2}-\mu_{t}\right)  \right]
}{\partial\vartheta}dt+\left[  \frac{\partial\left[  P_{2}\mu_{1}\right]
}{\partial\vartheta}-\frac{\partial P_{2}}{\partial\vartheta}X_{t}\right]
dt+\frac{\partial P_{3}}{\partial\vartheta}dX_{t}\\
&  =\frac{\partial P_{1}}{\partial\vartheta}\left(  \mu_{2}-\mu_{t}\right)
dt+P_{1}\left(  \frac{\partial\mu_{2}}{\partial\vartheta}-\frac{\partial
\mu_{t}}{\partial\vartheta}\right)  dt+\left[  P_{2}\frac{\partial\mu_{1}%
}{\partial\vartheta}+\frac{\partial P_{2}}{\partial\vartheta}\mu_{1}%
-\frac{\partial P_{2}}{\partial\vartheta}X_{t}\right]  dt\\
&  +\frac{\partial P_{3}}{\partial\vartheta}\left\{  {k_{11}}\left(  \mu
_{1}-X_{t}\right)  dt+{k_{12}}\left(  \mu_{2}-Y_{t}\right)  dt+{\sigma_{1}%
dW}_{1t}\right\} \\
&  =\frac{\partial P_{3}}{\partial\vartheta}{k_{12}}\left(  \mu_{2}%
-Y_{t}\right)  \cdot dt+\left(  \frac{\partial P_{3}}{\partial\vartheta
}{k_{11}+}\frac{\partial P_{2}}{\partial\vartheta}\right)  \left(  \mu
_{1}-X_{t}\right)  \cdot dt\\
&  +\frac{\partial P_{1}}{\partial\vartheta}\left(  \mu_{2}-\mu_{t}\right)
\cdot dt+P_{1}\left(  \left[  \frac{\partial\mu_{2}}{\partial\vartheta}%
+\frac{P_{2}}{P_{1}}\frac{\partial\mu_{1}}{\partial\vartheta}\right]
-\frac{\partial\mu_{t}}{\partial\vartheta}\right)  \cdot dt\\
&  +\frac{\partial P_{3}}{\partial\vartheta}{\sigma_{1}\cdot dW}_{1t}.
\end{align*}


Finally, we rewrite the dynamics of (note!!! $Y_{t}$ is in front of $X_{t}$)
\[
Z_{t}:=\left(  Y_{t},X_{t},\mu_{t},\frac{\partial\mu_{t}}{\partial\vartheta
}\right)
\]
as the OU SDE below%
\[
dZ_{t}=\Gamma(\mu_{Z}-Z_{t})dt+CdW_{t},
\]
where%
\[
W_{t}=\left[
\begin{array}
[c]{l}%
{W_{1,t}}\\
{W_{2,t}}%
\end{array}
\right]  ,
\]
and, the same with what is in codes,%
\[
\Gamma=\left[
\begin{array}
[c]{cccc}%
\Gamma_{11} & 0 & 0 & 0\\
\Gamma_{21} & \Gamma_{22} & 0 & 0\\
\Gamma_{31} & \Gamma_{32} & \Gamma_{33} & 0\\
\Gamma_{41} & \Gamma_{42} & \Gamma_{43} & \Gamma_{44}%
\end{array}
\right]  ,C=\left[
\begin{array}
[c]{cc}%
C_{11} & C_{12}\\
C_{21} & C_{22}\\
C_{31} & C_{32}\\
C_{41} & C_{42}%
\end{array}
\right]  ,\mu_{Z}=\left[
\begin{array}
[c]{c}%
\mu_{2}\\
\mu_{1}\\
\mu_{2}\\
\mu_{4}%
\end{array}
\right]
\]


In codes, mu is $\mu_{Z}$, v is $v^{\left(  0\right)  }$, P1 is $P_{1}$, P2 is
$P_{2}$, P3 is $P_{3}$. Other things in codes, like Part$\Gamma$ or PartC are
some intermediate things which help us to make up the final $\Gamma,\mu_{Z},C$.

Finally, using some matrix computations formula for OU model, like
KroneckerProduct, we obtain the stationary distribution of $Z_{t}:=\left(
Y_{t},X_{t},\mu_{t},\frac{\partial\mu_{t}}{\partial\vartheta}\right)  $ is, as
in codes,%
\[
Z_{\infty}\sim\text{Gaussian( Imean, Ivar ),}%
\]
where Imean$\in\mathbb{R}^{3+\dim(\vartheta)}$ and Ivar is a $\left[
3+\dim(\vartheta)\right]  \times\left[  3+\dim(\vartheta)\right]  $ matrix.

\subsection{Fisher information $I_{\vartheta}^{(1)}$ formulas}

\bigskip%
\begin{subequations}
\begin{align}
I_{\vartheta}^{(1)}  &  =I_{1,\alpha\beta\gamma}^{\text{M}}(\theta
_{0})=\mathbb{E}\left[  \frac{1}{\sigma_{11}^{2}(X;\zeta_{0})}\left(
\int_{\mathcal{Y}}\frac{\partial\mu_{1}}{\partial\vartheta}(X,\tilde{y}%
_{0};\alpha_{0})p(\tilde{y}_{0})d\tilde{y}_{0}+\int_{\mathcal{Y}}\mu
_{1}(X,\tilde{y}_{0};\alpha_{0})\mathfrak{p}_{\alpha\beta\gamma}(\tilde{y}%
_{0})d\tilde{y}_{0}\right)  \right. \label{SD information alpha_beta_gamma}\\
&  \left.  \times\left(  \int_{\mathcal{Y}}\frac{\partial\mu_{1}}%
{\partial\vartheta^{\intercal}}(X,\tilde{y}_{0};\alpha_{0})p(\tilde{y}%
_{0})d\tilde{y}_{0}+\int_{\mathcal{Y}}\mu_{1}(X,\tilde{y}_{0};\alpha
_{0})\mathfrak{p}_{\alpha\beta\gamma}^{\intercal}(\tilde{y}_{0})d\tilde{y}%
_{0}\right)  \right]  ,\nonumber
\end{align}
which we indeed have%
\end{subequations}
\[
\int_{\mathcal{Y}}\frac{\partial\mu_{1}}{\partial\vartheta}(X,\tilde{y}%
_{0};\alpha_{0})p(\tilde{y}_{0})d\tilde{y}_{0}+\int_{\mathcal{Y}}\mu
_{1}(X,\tilde{y}_{0};\alpha_{0})\mathfrak{p}_{\alpha\beta\gamma}(\tilde{y}%
_{0})d\tilde{y}_{0}=\frac{\partial}{\partial\vartheta}\left(  \int%
_{\mathcal{Y}}\mu_{1}(X,\tilde{y}_{0};\vartheta)p(\tilde{y}_{0};\vartheta
)d\tilde{y}_{0}\right)  .
\]


Now in our case --

drift function $\mu_{1}$:%

\[
k_{11}(x-\mu_{1})+k_{12}(y-\mu_{2})
\]


Fisher information $I_{\vartheta}^{(1)}$:%

\[
I_{\vartheta}^{(1)}=\mathbb{E}^{\pi_{0}}\left[  \frac{1}{\sigma_{11}^{2}%
}\left(  \frac{\partial}{\partial\vartheta}\left(  k_{11}(\mu_{1}%
-X_{t})+k_{12}\left(  \mu_{2}-\mu_{t}\right)  \right)  \right)  ^{2}\right]
\]
note that%
\begin{align*}
H  &  :=\frac{\partial}{\partial\vartheta}\left(  k_{11}(\mu_{1}-X_{t}%
)+k_{12}\left(  \mu_{2}-\mu_{t}\right)  \right) \\
&  =\frac{\partial k_{11}}{\partial\vartheta}\left(  \mu_{1}-X_{t}\right)
+k_{11}\frac{\partial\mu_{1}}{\partial\vartheta}+\frac{\partial k_{12}%
}{\partial\vartheta}\left(  \mu_{2}-\mu_{t}\right)  +k_{12}\left(
\frac{\partial\mu_{2}}{\partial\vartheta}-\frac{\partial\mu_{t}}%
{\partial\vartheta}\right) \\
&  =\frac{\partial k_{11}}{\partial\vartheta}\left(  \mu_{1}-X_{t}\right)
+\frac{\partial k_{12}}{\partial\vartheta}\left(  \mu_{2}-\mu_{t}\right)
-k_{12}\frac{\partial\mu_{t}}{\partial\vartheta}+\left(  k_{11}\frac
{\partial\mu_{1}}{\partial\vartheta}+k_{12}\frac{\partial\mu_{2}}%
{\partial\vartheta}\right)
\end{align*}
$H\in\mathbb{R}^{\dim(\vartheta)}$ is a Gaussian vector because it is the
linear combination of the Gaussian vector $Z_{\infty}$.

In codes, the mean of $H$ is%
\[
\text{Hmean}=E[H]=-k_{12}E\left[  \frac{\partial\mu_{t}}{\partial\vartheta
}\right]  +k_{11}\frac{\partial\mu_{1}}{\partial\vartheta}+k_{12}%
\frac{\partial\mu_{2}}{\partial\vartheta},
\]
the covariance matrix of $H$ is%
\[
\text{Hvar}=Cov[H]=\left[  \frac{\partial k_{11}}{\partial\vartheta}%
,\frac{\partial k_{12}}{\partial\vartheta},k_{12}I_{\dim(\vartheta)}\right]
\cdot\Sigma_{X,\mu,\frac{\partial\mu}{\partial\vartheta}}\cdot\left[
\begin{array}
[c]{c}%
\frac{\partial k_{11}}{\partial\vartheta},\\
\frac{\partial k_{12}}{\partial\vartheta},\\
k_{12}I_{\dim(\vartheta)}%
\end{array}
\right]  ,
\]
where $I_{\dim(\vartheta)}$ is the identity matrix with dimension
$\dim(\vartheta)$, and Hmean$\in\mathbb{R}^{\dim(\vartheta)}$ and Hvar is a
$\dim(\vartheta)\times\dim(\vartheta)$ matrix.

So we obtain the Fisher information is equal to%
\[
I_{\vartheta}^{(1)}=\frac{1}{\sigma_{11}^{2}}\left\{  Cov[H]+E[H]\cdot
(E[H])^{\top}\right\}  .
\]
In codes, FisherInformation and Fisher are both $I_{\vartheta}^{(1)}$, and we
save the formula in Fisher\_OurOUmodel.m.

For the model%
\begin{align*}
dX_{t}  &  =\kappa{_{1}({Y}_{t}-X_{t})}dt+\sigma_{1}dW_{1t},\\
dY_{t}  &  =\kappa{_{2}(\mu-{Y}_{t})}dt+\rho\sigma_{2}dW_{1t}+\sqrt{1-\rho
^{2}}\sigma_{2}dW_{2t},
\end{align*}
with%
\[
\vartheta=(\mu,\kappa{_{1},}\kappa_{2},\sigma_{2},\rho)^{\intercal}.
\]
In codes we verify that $I_{\vartheta}^{(1)}$ has the following structure%
\[
I_{\vartheta}^{(1)}=\left(
\begin{array}
[c]{ccccc}%
I_{\mu}^{(1)} & 0 & 0 & 0 & 0\\
0 &  &  &  & \\
0 &  &  &  & \\
0 &  &  &  & \\
0 &  &  &  &
\end{array}
\right)  \text{ where the right corner }4\times4\text{ submatrix has a very
long formula,}%
\]
and $I_{\vartheta}^{(1)}$ is full-rank, which implies that the whole parameter
vector $\vartheta=(\mu,\kappa{_{1},}\kappa_{2},\sigma_{2},\rho)^{\intercal}$
is identifiable.

However, according to Shengyi notes proof,
\[
\theta=(\sigma_{11},\vartheta)=(\sigma_{11},\mu,\kappa{_{1},}\kappa_{2}%
,\sigma_{2},\rho)^{\intercal}%
\]
should be unidentifiable. We will check this point, and re-run the AMMLE
simulation to see. Currently we will use the model with $\rho$ but do not
estimate $\rho$. Only the model with $\rho$ can help us to illustrate the
information loss of MIMLE v.s. FIMLE.

\bigskip

\bigskip

\bigskip



\subsection{math formulas foundation with codes explanations (has $\sigma_{1}$
v.s. $\sigma_{1,0}$ distinguish)}

We start by writing down the dynamics of all the processes we need.

By Kushner equation, (note that for $v_{t}[X_{[0,t]}^{\vartheta};\vartheta]$,
when we use $(X_{t}^{\vartheta},Y_{t}^{\vartheta})$ to establish its dynamics,
we will get the ODE below without any stochastic parts, so when we replace
$X_{[0,t]}^{\vartheta}$ by $X_{[0,t]}$, the dynamics of $v_{t}[X_{[0,t]}%
;\vartheta]$ is still the ODE below):

filter variance --%
\[
\frac{dv_{t}}{dt}=\sigma_{22}^{2}-\left(  2k_{22}-\frac{2k_{12}}{\sigma_{11}%
}\sigma_{21}\right)  v_{t}-\left(  \frac{k_{12}}{\sigma_{11}}\right)
^{2}v_{t}^{2}%
\]


filter mean -- for $\mu_{t}[X_{[0,t]};\vartheta]$:%
\begin{equation}
d\mu_{t}=P_{1}\left(  \mu s-\mu_{t}\right)  dt+P_{2}\left(  \mu_{1}%
-X_{t}\right)  dt+P_{3}dX_{t},
\end{equation}
where $P_{1},P_{2}$ and $P_{3}$ are functions of parameters%
\[
P_{1}=k_{22}-k_{12}\left[  -\frac{k_{12}v^{\left(  0\right)  }}{\sigma_{1}%
^{2}}+\frac{\sigma_{21}}{\sigma_{1}}\right]  ,P_{2}=-k_{11}\left[
-\frac{k_{12}v^{\left(  0\right)  }}{\sigma_{1}^{2}}+\frac{\sigma_{21}}%
{\sigma_{1}}\right]  ,P_{3}=\left[  -\frac{k_{12}v^{\left(  0\right)  }%
}{\sigma_{1}^{2}}+\frac{\sigma_{21}}{\sigma_{1}}\right]  ,
\]
where for simplicity of computation we directly plug the stationary value
$v^{\left(  0\right)  }$ of $v_{t}$ into the above dynamics of $\mu_{t}$.

Then we obtain (here $\mu_{t}$ represents $\mu_{t}[X_{[0,t]};\vartheta]$)
\begin{align*}
d\left[  \frac{\partial\mu_{t}}{\partial\vartheta}\right]   &  =\frac
{\partial\left[  P_{1}\left(  \mu s-\mu_{t}\right)  \right]  }{\partial
\vartheta}dt+\frac{\partial\left[  P_{2}\left(  \mu_{1}-X_{t}\right)  \right]
}{\partial\vartheta}dt+\frac{\partial P_{3}}{\partial\vartheta}dX_{t}\\
&  =\frac{\partial\left[  P_{1}\left(  \mu s-\mu_{t}\right)  \right]
}{\partial\vartheta}dt+\left[  \frac{\partial\left[  P_{2}\mu_{1}\right]
}{\partial\vartheta}-\frac{\partial P_{2}}{\partial\vartheta}X_{t}\right]
dt+\frac{\partial P_{3}}{\partial\vartheta}dX_{t}\\
&  =\frac{\partial P_{1}}{\partial\vartheta}\left(  \mu s-\mu_{t}\right)
dt+P_{1}\left(  \frac{\partial\mu s}{\partial\vartheta}-\frac{\partial\mu_{t}%
}{\partial\vartheta}\right)  dt+\left[  P_{2}\frac{\partial\mu_{1}}%
{\partial\vartheta}+\frac{\partial P_{2}}{\partial\vartheta}\mu_{1}%
-\frac{\partial P_{2}}{\partial\vartheta}X_{t}\right]  dt\\
&  +\frac{\partial P_{3}}{\partial\vartheta}\left\{  {k_{11}}\left(  \mu
_{1}-X_{t}\right)  dt+{k_{12}}\left(  \mu_{2}-Y_{t}\right)  dt+{\sigma_{1}%
dW}_{1t}\right\}  \\
&  =\frac{\partial P_{3}}{\partial\vartheta}{k_{12}}\left(  \mu_{2}%
-Y_{t}\right)  \cdot dt+\left(  \frac{\partial P_{3}}{\partial\vartheta
}{k_{11}+}\frac{\partial P_{2}}{\partial\vartheta}\right)  \left(  \mu
_{1}-X_{t}\right)  \cdot dt\\
&  +\frac{\partial P_{1}}{\partial\vartheta}\left(  \mu s-\mu_{t}\right)
\cdot dt+P_{1}\left(  \left[  \frac{\partial\mu s}{\partial\vartheta}%
+\frac{P_{2}}{P_{1}}\frac{\partial\mu_{1}}{\partial\vartheta}\right]
-\frac{\partial\mu_{t}}{\partial\vartheta}\right)  \cdot dt\\
&  +\frac{\partial P_{3}}{\partial\vartheta}{\sigma_{1}\cdot dW}_{1t}.
\end{align*}


Finally, we rewrite the dynamics of (note!!! $Y_{t}$ is in front of $X_{t}$)
\[
Z_{t}:=\left(  Y_{t},X_{t},\mu_{t},\frac{\partial\mu_{t}}{\partial\vartheta
}\right)
\]
as the OU SDE below%
\[
dZ_{t}=\Gamma(\mu_{Z}-Z_{t})dt+CdW_{t},
\]
where%
\[
W_{t}=\left[
\begin{array}
[c]{l}%
{W_{1,t}}\\
{W_{2,t}}%
\end{array}
\right]  ,
\]
and, the same with what is in codes,%
\[
\Gamma=\left[
\begin{array}
[c]{cccc}%
\Gamma_{11} & 0 & 0 & 0\\
\Gamma_{21} & \Gamma_{22} & 0 & 0\\
\Gamma_{31} & \Gamma_{32} & \Gamma_{33} & 0\\
\Gamma_{41} & \Gamma_{42} & \Gamma_{43} & \Gamma_{44}%
\end{array}
\right]  ,C=\left[
\begin{array}
[c]{cc}%
C_{11} & C_{12}\\
C_{21} & C_{22}\\
C_{31} & C_{32}\\
C_{41} & C_{42}%
\end{array}
\right]  ,\mu_{Z}=\left[
\begin{array}
[c]{c}%
\mu_{2}\\
\mu_{1}\\
\mu s\\
\mu_{4}%
\end{array}
\right]
\]


In codes, mu is $\mu_{Z}$, v is $v^{\left(  0\right)  }$, P1 is $P_{1}$, P2 is
$P_{2}$, P3 is $P_{3}$. Other things in codes, like Part$\Gamma$ or PartC are
some intermediate things which help us to make up the final $\Gamma,\mu_{Z},C$.

Finally, using some matrix computations formula for OU model, like
KroneckerProduct, we obtain the stationary distribution of $Z_{t}:=\left(
Y_{t},X_{t},\mu_{t},\frac{\partial\mu_{t}}{\partial\vartheta}\right)  $ is, as
in codes,%
\[
Z_{\infty}\sim\text{Gaussian( Imean, Ivar ),}%
\]
where Imean$\in\mathbb{R}^{3+\dim(\vartheta)}$ and Ivar is a $\left[
3+\dim(\vartheta)\right]  \times\left[  3+\dim(\vartheta)\right]  $ matrix.



\bigskip

\section{$I_{\alpha\beta,\gamma}$}

\bigskip

\bigskip

\bigskip for the diagonal block $\mathcal{I}_{\alpha\beta}^{\text{M}}%
(\Delta,\xi_{0})$ and the off-diagonal blocks $\mathcal{I}_{\alpha\beta
,\gamma}^{\text{M}}(\Delta,\xi_{0})\equiv(\mathcal{I}_{\gamma,\alpha\beta
}^{\text{M}}(\Delta,\xi_{0}))^{\top}$, their leading order expansion
coefficients $\mathcal{I}_{0,\alpha\beta}^{\text{M}}(\xi_{0})$ and
$\mathcal{I}_{0,\alpha\beta,\gamma}^{\text{M}}(\xi_{0})$ are zero, and their
lowest nonzero orders in $\Delta$ are $O(\Delta)$, i.e.,%
\[
\mathcal{I}_{\alpha\beta}^{\text{M}}(\Delta,\xi_{0})=\mathcal{I}%
_{1,\alpha\beta}^{\text{M}}(\xi_{0})\Delta+o(\Delta)\text{ and }%
\mathcal{I}_{\alpha\beta,\gamma}^{\text{M}}(\Delta,\xi_{0})=\mathcal{I}%
_{1,\alpha\beta,\gamma}^{\text{M}}(\xi_{0})\Delta+o(\Delta),
\]
where, with $\vartheta=(\alpha^{\intercal},\beta^{\intercal})^{\intercal}$,
the first order expansion coefficient $\mathcal{I}_{1,\alpha\beta}^{\text{M}%
}(\xi_{0})$ of $\mathcal{I}_{\alpha\beta}^{\text{M}}(\Delta,\xi_{0})$ is given
by
\begin{align}
\mathcal{I}_{1,\alpha\beta}^{\text{M}}(\xi_{0}) &  =\frac{1}{\sigma_{11}%
^{2}(x_{0};\gamma_{0})}\left(  \int_{\mathcal{Y}}\frac{\partial\mu_{1}%
}{\partial\vartheta}(x_{0},\tilde{y}_{0};\alpha_{0})f_{0}(\tilde{y}%
_{0})d\tilde{y}_{0}+\int_{\mathcal{Y}}\mu_{1}(x_{0},\tilde{y}_{0};\alpha
_{0})g_{0,\alpha\beta}(\tilde{y}_{0})d\tilde{y}_{0}\right)
\label{I_M_1 ab SD}\\
&  \cdot\left(  \int_{\mathcal{Y}}\frac{\partial\mu_{1}}{\partial\vartheta
}(x_{0},\tilde{y}_{0};\alpha_{0})f_{0}(\tilde{y}_{0})d\tilde{y}_{0}%
+\int_{\mathcal{Y}}\mu_{1}(x_{0},\tilde{y}_{0};\alpha_{0})g_{0,\alpha\beta
}(\tilde{y}_{0})d\tilde{y}_{0}\right)  ^{\intercal},\nonumber
\end{align}
and the first order expansion coefficient $\mathcal{I}_{1,\alpha\beta,\gamma
}^{\text{M}}(\xi_{0})$ of $\mathcal{I}_{\alpha\beta,\gamma}^{\text{M}}%
(\Delta,\xi_{0})$ is given by%
\begin{align}
\mathcal{I}_{1,\alpha\beta,\gamma}^{\text{M}}(\xi_{0}) &  =\frac{1}%
{\sigma_{11}^{2}(x_{0};\gamma_{0})}\left(  \int_{\mathcal{Y}}\frac{\partial
\mu_{1}}{\partial\vartheta}(x_{0},\tilde{y}_{0};\alpha_{0})f_{0}(\tilde{y}%
_{0})d\tilde{y}_{0}+\int_{\mathcal{Y}}\mu_{1}(x_{0},\tilde{y}_{0};\alpha
_{0})g_{0,\alpha\beta}(\tilde{y}_{0})d\tilde{y}_{0}\right)
\label{I_M_1 ab c SD}\\
&  \cdot\left(  \int_{\mathcal{Y}}\mu_{1}(x_{0},\tilde{y}_{0};\alpha
_{0})g_{0,\gamma}(\tilde{y}_{0})d\tilde{y}_{0}-\left(
\begin{array}
[c]{c}%
4\int_{\mathcal{Y}}\frac{\mu_{1}(x_{0},\tilde{y}_{0};\alpha_{0})}{\sigma
_{11}(x_{0};\gamma_{0})}f_{0}(\tilde{y}_{0})d\tilde{y}_{0}+2\frac
{\partial\sigma_{11}}{\partial x_{0}}(x_{0};\gamma_{0})\\
-2\frac{\mu_{1}(x_{0},y_{0};\alpha_{0})}{\sigma_{11}(x_{0};\gamma_{0})}%
\end{array}
\right)  \frac{\partial\sigma_{11}}{\partial\gamma}(x_{0};\gamma_{0})\right)
^{\intercal}\nonumber\\
&  +\left(  2\int_{\mathcal{Y}}\mu_{1}(x_{0},\tilde{y}_{0};\alpha_{0}%
)\frac{\partial\mu_{1}}{\partial\vartheta}(x_{0},\tilde{y}_{0};\alpha
_{0})f_{0}(\tilde{y}_{0})d\tilde{y}_{0}+\int_{\mathcal{Y}}[\mu_{1}%
(x_{0},\tilde{y}_{0};\alpha_{0})]^{2}g_{0,\alpha\beta}(\tilde{y}_{0}%
)d\tilde{y}_{0}\right)  \left(  \frac{\frac{\partial\sigma_{11}}%
{\partial\gamma}(x_{0};\gamma_{0})}{\sigma_{11}^{3}(x_{0};\gamma_{0})}\right)
^{\intercal}\nonumber\\
&  +\left(  \int_{\mathcal{Y}}\frac{\partial^{2}\mu_{1}}{\partial
x_{0}\partial\vartheta}(x_{0},\tilde{y}_{0};\alpha_{0})f_{0}(\tilde{y}%
_{0})d\tilde{y}_{0}+\int_{\mathcal{Y}}\frac{\partial\mu_{1}}{\partial x_{0}%
}(x_{0},\tilde{y}_{0};\alpha_{0})g_{0,\alpha\beta}(\tilde{y}_{0})d\tilde
{y}_{0}\right)  \left(  \frac{\frac{\partial\sigma_{11}}{\partial\gamma}%
(x_{0};\gamma_{0})}{\sigma_{11}(x_{0};\gamma_{0})}\right)  ^{\intercal}\\
&  +\left(
\begin{array}
[c]{c}%
\int_{\mathcal{Y}}\frac{\partial\sigma_{21}}{\partial\vartheta}(x_{0}%
,\tilde{y}_{0};\beta_{0})\frac{\partial\mu_{1}}{\partial\tilde{y}_{0}}%
(x_{0},\tilde{y}_{0};\alpha_{0})f_{0}(\tilde{y}_{0})d\tilde{y}_{0}\\
+\int_{\mathcal{Y}}\sigma_{21}(x_{0},\tilde{y}_{0};\beta_{0})\frac
{\partial^{2}\mu_{1}}{\partial\tilde{y}_{0}\partial\vartheta}(x_{0},\tilde
{y}_{0};\alpha_{0})f_{0}(\tilde{y}_{0})d\tilde{y}_{0}\\
+\int_{\mathcal{Y}}\sigma_{21}(x_{0},\tilde{y}_{0};\beta_{0})\frac{\partial
\mu_{1}}{\partial\tilde{y}_{0}}(x_{0},\tilde{y}_{0};\alpha_{0})g_{0,\alpha
\beta}(\tilde{y}_{0})d\tilde{y}_{0}%
\end{array}
\right)  \left(  \frac{\frac{\partial\sigma_{11}}{\partial\gamma}(x_{0}%
;\gamma_{0})}{\sigma_{11}^{2}(x_{0};\gamma_{0})}\right)  ^{\intercal}\nonumber
\end{align}
for the diagonal block $\mathcal{I}_{\gamma}^{\text{M}}(\Delta,\xi_{0})$, its
lowest nonzero order in $\Delta$ is $O(1)$, i.e.,%
\[
\mathcal{I}_{\gamma}^{\text{M}}(\Delta,\xi_{0})=\mathcal{I}_{0,\gamma
}^{\text{M}}(\xi_{0})+\mathcal{I}_{1,\gamma}^{\text{M}}(\xi_{0})\Delta
+o(\Delta),
\]
where, the leading order expansion coefficient $\mathcal{I}_{0,\gamma
}^{\text{M}}(\xi_{0})$ of $\mathcal{I}_{\gamma}^{\text{M}}(\Delta,\xi_{0})$ is
given by \emph{ }%
\begin{subequations}
\begin{equation}
\mathcal{I}_{0,\gamma}^{\text{M}}(\xi_{0})=\frac{2}{\sigma_{11}^{2}%
(x_{0};\gamma_{0})}\frac{\partial\sigma_{11}}{\partial\gamma}(x_{0};\gamma
_{0})\frac{\partial\sigma_{11}}{\partial\gamma^{\intercal}}(x_{0};\gamma
_{0}),\label{I_M_0 SD}%
\end{equation}
and the first order expansion coefficient $\mathcal{I}_{1,\gamma}^{\text{M}%
}(\xi_{0})$ of $\mathcal{I}_{\gamma}^{\text{M}}(\Delta,\xi_{0})$ is lengthy so
we put its formula in ??? in Section
\ref{proof of prop: one-step expansion I_M} of the online supplementary
material. \emph{]}

\bigskip

\bigskip

\subsection{\bigskip}

$\mu_{1}(x,y;\alpha)=\kappa{_{1}(y-x)}${,} $\mu_{2}(y;\alpha)=\kappa{_{2}%
(\mu-y)}${,} $\sigma_{11}(x;\gamma)=\sigma_{1}$, $\sigma_{21}(y;\beta
)=\rho\sigma_{2}$, and $\sigma_{22}(y;\beta)=\sqrt{1-\rho^{2}}\sigma_{2}$



\bigskip In BOU model, we have%
\end{subequations}
\begin{align}
\mathcal{I}_{1,\alpha\beta,\gamma}^{\text{M}}(\xi_{0}) &  =\left(
\frac{\partial}{\partial\vartheta}\int_{\mathcal{Y}}\mu_{1}(x_{0},\tilde
{y}_{0};\alpha_{0})p(\tilde{y}_{0};\theta)d\tilde{y}_{0}\right)  \\
&  \cdot\frac{1}{\sigma_{1}^{2}}\left(  \int_{\mathcal{Y}}\mu_{1}(x_{0}%
,\tilde{y}_{0};\alpha_{0})g_{0,\gamma}(\tilde{y}_{0})d\tilde{y}_{0}-\left(
4\int_{\mathcal{Y}}\frac{\mu_{1}(x_{0},\tilde{y}_{0};\alpha_{0})}{\sigma_{1}%
}f_{0}(\tilde{y}_{0})d\tilde{y}_{0}-2\frac{\mu_{1}(x_{0},y_{0};\alpha_{0}%
)}{\sigma_{1}}\right)  \right)  ^{\intercal}\nonumber\\
&  +\left(  \frac{\partial}{\partial\vartheta}\int_{\mathcal{Y}}[\mu_{1}%
(x_{0},\tilde{y}_{0};\alpha_{0})]^{2}p(\tilde{y}_{0};\theta)d\tilde{y}%
_{0}\right)  \frac{1}{\sigma_{1}^{3}}\nonumber\\
&  +\left(  \frac{\partial}{\partial\vartheta}\int_{\mathcal{Y}}\frac
{\partial\mu_{1}}{\partial x_{0}}(x_{0},\tilde{y}_{0};\alpha)p(\tilde{y}%
_{0};\theta)d\tilde{y}_{0}\right)  \frac{1}{\sigma_{1}}\\
&  +\left(  \frac{\partial}{\partial\vartheta}\int_{\mathcal{Y}}\rho\sigma
_{2}\frac{\partial\mu_{1}}{\partial\tilde{y}_{0}}(x_{0},\tilde{y}_{0}%
;\alpha_{0})p(\tilde{y}_{0};\theta)d\tilde{y}_{0}\right)  \frac{1}{\sigma
_{1}^{2}}\nonumber
\end{align}%
\begin{align}
I_{1,\alpha\beta,\gamma}^{\text{M}} &  =\left(  \frac{\partial}{\partial
\vartheta}\left[  \kappa{_{1}(\mu(\theta)-X)}\right]  \right)  \\
&  \cdot\frac{1}{\sigma_{1}^{2}}\left(  \frac{\partial}{\partial\sigma_{1}%
}\left[  \kappa{_{1}(\mu(\theta)-X)}\right]  -\left(  4\frac{\kappa{_{1}%
(\mu(\theta)-X)}}{\sigma_{1}}-2\frac{\kappa{_{1}(Y-X)}}{\sigma_{1}}\right)
\right)  ^{\intercal}\nonumber\\
&  +\left(  \frac{\partial}{\partial\vartheta}\int_{\mathcal{Y}}[\kappa
{_{1}(\tilde{y}_{0}-X)}]^{2}p(\tilde{y}_{0};\theta)d\tilde{y}_{0}\right)
\frac{1}{\sigma_{1}^{3}}\nonumber\\
&  +\left(  \frac{\partial}{\partial\vartheta}\int_{\mathcal{Y}}[-\kappa
_{1}]p(\tilde{y}_{0};\theta)d\tilde{y}_{0}\right)  \frac{1}{\sigma_{1}}\\
&  +\left(  \frac{\partial}{\partial\vartheta}\int_{\mathcal{Y}}\rho\sigma
_{2}[\kappa_{1}]p(\tilde{y}_{0};\theta)d\tilde{y}_{0}\right)  \frac{1}%
{\sigma_{1}^{2}}\nonumber
\end{align}
so we get%
\begin{align}
I_{1,\alpha\beta,\gamma}^{\text{M}} &  =\left(  \frac{\partial\kappa{_{1}}%
}{\partial\vartheta}\left[  {\mu(\theta)-X}\right]  +\kappa{_{1}}%
\frac{\partial{\mu(\theta)}}{\partial\vartheta}\right)  \\
&  \cdot\frac{\kappa{_{1}}}{\sigma_{1}^{2}}\left(  \frac{\partial{\mu(\theta
)}}{\partial\sigma_{1}}-\left(  4\frac{{\mu(\theta)-X}}{\sigma_{1}}%
-2\frac{{Y-X}}{\sigma_{1}}\right)  \right)  ^{\intercal}\nonumber\\
&  +\left(  \frac{\partial}{\partial\vartheta}\left[  (\kappa_{1})^{2}\left[
{(\mu(\theta)-X)}^{2}+v{(\theta)}\right]  \right]  \right)  \frac{1}%
{\sigma_{1}^{3}}\nonumber\\
&  +\left(  \frac{\partial}{\partial\vartheta}[-\kappa_{1}]\right)  \frac
{1}{\sigma_{1}}+\left(  \frac{\partial}{\partial\vartheta}\left[  \kappa
_{1}\rho\sigma_{2}\right]  \right)  \frac{1}{\sigma_{1}^{2}}%
\end{align}%
\begin{align*}
I_{1,\alpha\beta,\gamma}^{\text{M}}  & =\frac{2\kappa{_{1}}}{\sigma_{1}^{3}%
}E\left[  \left(  \frac{\partial\kappa{_{1}}}{\partial\vartheta}\left[
{-X+\mu(\theta)}\right]  +\kappa{_{1}}\frac{\partial{\mu(\theta)}}%
{\partial\vartheta}\right)  \left(  X+Y-2{\mu(\theta)}+\frac{\sigma_{1}}%
{2}\frac{\partial{\mu(\theta)}}{\partial\sigma_{1}}\right)  \right]  \\
& +\frac{1}{\sigma_{1}^{3}}E\left[  \frac{\partial(\kappa_{1})^{2}}%
{\partial\vartheta}\left[  {(\mu(\theta)-X)}^{2}\right]  +2(\kappa_{1}%
)^{2}\frac{\partial{\mu(\theta)}}{\partial\vartheta}{(\mu(\theta)-X)}%
+\frac{\partial}{\partial\vartheta}\left[  (\kappa_{1})^{2}v{(\theta)}\right]
\right]  \\
& -\frac{1}{\sigma_{1}}\left(  \frac{\partial}{\partial\vartheta}[\kappa
_{1}]\right)  +\frac{1}{\sigma_{1}^{2}}\left(  \frac{\partial}{\partial
\vartheta}\left[  \kappa_{1}\rho\sigma_{2}\right]  \right)
\end{align*}
we are left to do the terms:%
\[
\frac{1}{\sigma_{1}^{3}}E\left[  \frac{\partial(\kappa_{1})^{2}}%
{\partial\vartheta}\left[  {(\mu(\theta)-X)}^{2}\right]  +2(\kappa_{1}%
)^{2}\frac{\partial{\mu(\theta)}}{\partial\vartheta}{(\mu(\theta)-X)}\right]
\]


\bigskip

\subsection{problem}

\bigskip

FI0 is given by%
\[
\frac{\left[  (X_{\Delta}-X_{0})^{2}-\sigma_{11}^{2}(X_{0};\theta_{0}%
)\Delta\right]  ^{2}}{\sigma_{11}^{4}(X_{0};\theta_{0})\Delta^{2}}\cdot
\frac{\frac{\partial\sigma_{11}}{\partial\theta}\frac{\partial\sigma_{11}%
}{\partial\theta^{\intercal}}(X_{0};\theta_{0})}{\sigma_{11}^{2}(X_{0}%
;\theta_{0})}%
\]
and its expectation will indeed contribute to $O(\Delta)$-order expansion
coeffcient as well:

Indeed, applying Dynkin expansion to%
\[
\frac{1}{\Delta^{2}}\left[  (X_{\Delta}-X_{0})^{2}-\sigma_{11}^{2}%
(X_{0};\theta_{0})\Delta\right]  ^{2}=\frac{1}{\Delta^{2}}(x-X_{0})^{4}%
-\frac{1}{\Delta}2(x-X_{0})^{2}\sigma_{11}^{2}(X_{0};\theta_{0})+\sigma
_{11}^{4}(X_{0};\theta_{0}),
\]
we obtain%
\[
\]
for a sufficiently smooth function $\phi(\Delta,x,y)$ defined on
$[0,+\infty)\times\mathcal{X}\times\mathcal{Y}$,%
\begin{equation}
\mathbb{E}\left[  \phi(\Delta,X_{\Delta},Y_{\Delta})|(X_{0},Y_{0}%
)=(x_{0},y_{0})\right]  =\sum_{k=0}^{K}\mathcal{A}_{0}^{k}\phi(0,x_{0}%
,y_{0})\Delta^{k}+o(\Delta^{K}), \label{iterated Dynkin formula}%
\end{equation}
for any arbitrary order $K$, where the differential operator $\mathcal{A}_{0}$
is defined by%
\begin{align}
\mathcal{A}_{0}  &  :=\frac{\partial}{\partial\Delta}+\mu_{1}\left(
x,y;\theta_{0}\right)  \frac{\partial}{\partial x}+\mu_{2}\left(
x,y;\theta_{0}\right)  \frac{\partial}{\partial y}+\frac{1}{2}\sigma_{11}%
^{2}\left(  x,y;\theta_{0}\right)  \frac{\partial^{2}}{\partial x^{2}%
}\nonumber\\
&  +\sigma_{11}\left(  x,y;\theta_{0}\right)  \sigma_{21}\left(
x,y;\theta_{0}\right)  \frac{\partial^{2}}{\partial x\partial y}+\frac{1}%
{2}\left[  \sigma_{21}^{2}\left(  x,y;\theta_{0}\right)  +\sigma_{22}%
^{2}\left(  x,y;\theta_{0}\right)  \right]  \frac{\partial^{2}}{\partial
y^{2}}, \label{A0_operator}%
\end{align}
with $\mathcal{A}_{0}^{0}\phi=\phi$ and $\mathcal{A}_{0}^{k+1}\phi
=\mathcal{A}_{0}(\mathcal{A}_{0}^{k}\phi)$.

So we need to consider%
\begin{align*}
&  \mathcal{A}_{0}(\mathcal{A}_{0}(\mathcal{A}_{0}(x-x_{0})^{4}))\\
&  =\mathcal{A}_{0}(\mathcal{A}_{0}(\left[  \mu_{1}\left(  x,y;\theta
_{0}\right)  \frac{\partial}{\partial x}+\frac{1}{2}\sigma_{11}^{2}\left(
x;\theta_{0}\right)  \frac{\partial^{2}}{\partial x^{2}}\right]  (x-x_{0}%
)^{4}))\\
&  =\mathcal{A}_{0}(\mathcal{A}_{0}(4\mu_{1}\left(  x,y;\theta_{0}\right)
(x-x_{0})^{3}+6\sigma_{11}^{2}\left(  x;\theta_{0}\right)  (x-x_{0})^{2}))
\end{align*}


\bigskip

\bigskip

\bigskip

\section{FIMLE Fisher information from \textquotedblleft
Infill\_likelihood\_0211 notes\textquotedblright\ (no need these)}

\bigskip

\bigskip Consider the two dimensional Ornstein-Uhlenbeck (2DOU) processes,
\[
d\left[
\begin{array}
[c]{l}%
{X_{1}(t)}\\
{X_{2}(t)}%
\end{array}
\right]  =\left[
\begin{array}
[c]{ll}%
{k_{11}} & {0}\\
{k_{21}} & {k_{22}}%
\end{array}
\right]  \left[
\begin{array}
[c]{l}%
{\mu_{1}-X_{1}(t)}\\
{\mu_{2}-X_{2}(t)}%
\end{array}
\right]  dt+\left[
\begin{array}
[c]{cc}%
{\sigma_{1}} & {0}\\
{\sigma_{2}\rho} & {\sigma_{2}\sqrt{1-\rho^{2}}}%
\end{array}
\right]  d\left[
\begin{array}
[c]{l}%
{W_{1}(t)}\\
{W_{2}(t)}%
\end{array}
\right]
\]
where $\alpha=\left(  k_{11},k_{21},k_{22},\mu_{1},\mu_{2}\right)  ^{\top}$
and $\beta=\left(  \sigma_{1},\sigma_{2},\rho\right)  ^{\top}$ are parameters
respectively in drift and diffusion terms.

[[[

to transfer this model to our model, say%
\[
d\left[
\begin{array}
[c]{l}%
{X}_{t}\\
{Y}_{t}%
\end{array}
\right]  =\left[
\begin{array}
[c]{ll}%
{k_{11}} & {k_{12}}\\
{0} & {k_{22}}%
\end{array}
\right]  \left[
\begin{array}
[c]{l}%
{\mu_{1}-X}_{t}\\
{\mu_{2}-Y}_{t}%
\end{array}
\right]  dt+\left[
\begin{array}
[c]{cc}%
{\sigma_{1}} & {0}\\
{\sigma_{2}}\rho & {\sigma_{2}\sqrt{1-\rho^{2}}}%
\end{array}
\right]  d\left[
\begin{array}
[c]{l}%
{W_{1,t}}\\
{W_{2,t}}%
\end{array}
\right]  ,
\]
with ${\sigma_{11}=\sigma_{1},}$ ${\sigma_{21}=\sigma_{2}}\rho,$ ${\sigma
_{22}=\sigma_{2}\sqrt{1-\rho^{2}}}$.

The special OU model we are using is obtained by $\kappa_{11}=-\kappa
_{12}=\kappa_{1}$, $\kappa_{22}=\kappa_{2}$, $\mu_{1}=\mu_{2}=\mu$.

\bigskip We need to change the diffusion matrix to $\left[
\begin{array}
[c]{cc}%
{\sigma_{2}\rho} & {\sigma_{2}\sqrt{1-\rho^{2}}}\\
\sigma_{1} &
\end{array}
\right]  $ and regard ${X_{1}(t)}$ as $Y_{t}$, ${X_{2}(t)}$ as $X_{t}$,
$k_{11}$ as $\kappa_{2}$, $k_{22}=-k_{21}$ as $\kappa_{1}$.

For the Fisher information of drift parameters: The action on $k_{22},k_{21}$
requires us to add up the information of them by compute the quadratics
$(1,-1)I_{k_{22},k_{21}}(1,-1)^{\intercal}$. The action on $\mu_{1},\mu_{2}$
requires us to add up the information of them by compute the quadratics
$(1,1)I_{\mu_{1},\mu_{2}}(1,1)^{\intercal}$. The action on $\left[
\begin{array}
[c]{cc}%
{\sigma_{2}\rho} & {\sigma_{2}\sqrt{1-\rho^{2}}}\\
\sigma_{1} &
\end{array}
\right]  $ needs us to reverse the order of $\sigma_{1}$ and $\sigma_{2}$ in
the information of drift parameters because
\[
\left[
\begin{array}
[c]{cc}%
{\sigma_{2}\rho} & {\sigma_{2}\sqrt{1-\rho^{2}}}\\
\sigma_{1} &
\end{array}
\right]  \left[
\begin{array}
[c]{cc}%
{\sigma_{2}\rho} & {\sigma_{2}\sqrt{1-\rho^{2}}}\\
\sigma_{1} &
\end{array}
\right]  ^{\intercal}=\left[
\begin{array}
[c]{cc}%
{\sigma_{2}^{2}} & {\rho\sigma_{1}\sigma_{2}}\\
{\rho\sigma_{1}\sigma_{2}} & {\sigma_{1}^{2}}%
\end{array}
\right]  \text{ not }\left[
\begin{array}
[c]{cc}%
{\sigma_{1}^{2}} & {\rho\sigma_{1}\sigma_{2}}\\
{\rho\sigma_{1}\sigma_{2}} & {\sigma_{2}^{2}}%
\end{array}
\right]
\]


For the Fisher information of diffusion parameters: no action is need.

]]]

The corresponding terms in our general model setting are%
\[
\mu\left(  X\left(  t\right)  ;\alpha\right)  =\left[
\begin{array}
[c]{l}%
{k_{11}}\left(  {\mu_{1}-X_{1}(t)}\right) \\
{k_{21}}\left(  {\mu_{1}-X_{1}(t)}\right)  +{k_{22}}\left(  {\mu_{2}-X_{2}%
(t)}\right)
\end{array}
\right]  ,
\]
and%
\begin{align*}
\sigma\left(  X\left(  t\right)  ;\beta\right)   &  =\left[
\begin{array}
[c]{cc}%
{\sigma_{1}} & {0}\\
{\sigma_{2}\rho} & {\sigma_{2}\sqrt{1-\rho^{2}}}%
\end{array}
\right]  ,\\
a\left(  X\left(  t\right)  ;\beta\right)   &  =\left[
\begin{array}
[c]{cc}%
{\sigma_{1}^{2}} & {\rho\sigma_{1}\sigma_{2}}\\
{\rho\sigma_{1}\sigma_{2}} & {\sigma_{2}^{2}}%
\end{array}
\right]  .
\end{align*}
Note that in this case, $\sigma\left(  X\left(  t\right)  ;\beta\right)  $ and
$a\left(  X\left(  t\right)  ;\beta\right)  $ are independent of $X\left(
t\right)  $. Plug the expressions above into Theorem
\ref{thm:LDasymptotics_result}, i.e.%
\[
\sqrt{T}\left(  \widehat{\alpha}-\alpha_{0}\right)  \rightarrow_{d}N\left(
0,\left\{  \mathbb{E}^{\pi}\left[  \left(  \frac{\partial\widetilde{\mu}%
}{\partial\alpha}\right)  ^{\top}\widetilde{\sigma}^{-T}\widetilde{\sigma
}^{-1}\frac{\partial\widetilde{\mu}}{\partial\alpha}\right]  \right\}
^{-1}\right)  ,
\]%
\[
\sqrt{\frac{T}{\Delta}}\left(  \widehat{\beta}-\beta_{0}\right)
\rightarrow_{d}N\left(  0,\left\{  \mathbb{E}^{\pi}\left[  -\frac{1}{2}%
\sum_{k,l=1}^{d}\left(  \frac{\partial\widetilde{a}_{kl}}{\partial\beta
}\right)  ^{\top}\frac{\partial\widetilde{a}_{kl}^{-1}}{\partial\beta}\right]
\right\}  ^{-1}\right)  ,
\]
where $\pi$ is the stationary distribution of $X\left(  t\right)  $, we can
get the asymptotic distribution of MLE $\widehat{\alpha}$ and $\widehat{\beta
}$. Then by the stationary distribution of $\left(  X_{1}(t),X_{2}(t)\right)
$, i.e.
\[
N\left(  \left[
\begin{array}
[c]{c}%
{\mu}_{1}\\
{\mu}_{2}%
\end{array}
\right]  ,\left[
\begin{array}
[c]{cc}%
{\frac{\sigma_{1}^{2}}{2k_{11}}} & {-\frac{k_{21}\sigma_{1}^{2}-2\rho
k_{11}\sigma_{1}\sigma_{2}}{2k_{11}\left(  k_{11}+k_{22}\right)  }}\\
{-\frac{k_{21}\sigma_{1}^{2}-2\rho k_{11}\sigma_{1}\sigma_{2}}{2k_{11}\left(
k_{11}+k_{22}\right)  }} & {\frac{k_{21}^{2}\sigma_{1}^{2}-2k_{11}k_{21}%
\rho\sigma_{1}\sigma_{2}+k_{11}\left(  k_{11}+k_{22}\right)  \sigma_{2}^{2}%
}{2k_{11}k_{22}\left(  k_{11}+k_{22}\right)  }}%
\end{array}
\right]  \right)  ,
\]
we can plug it into the two distributions expression above, and get the
explicit expression of asymptotic covariance matrix of MLE $\widehat{\alpha}$
and $\widehat{\beta}$ respectively based on the real value of parameters
$\alpha_{0}$ and $\beta_{0}$. Hereafter we directly use $k_{11},k_{21}$,etc,
to represent the real value of parameters, and denote the scaled Fisher
information based on Theorem \ref{thm:LDasymptotics_result} as$\ $%
\[
I^{\left(  1\right)  }=\mathbb{E}^{\pi}\left[  \left(  \frac{\partial
\widetilde{\mu}}{\partial\alpha}\right)  ^{\top}\widetilde{\sigma}%
^{-T}\widetilde{\sigma}^{-1}\frac{\partial\widetilde{\mu}}{\partial\alpha
}\right]  ,
\]
$\ $and$\ $%
\[
I^{\left(  2\right)  }=\mathbb{E}^{\pi}\left[  -\frac{1}{2}\sum_{k,l=1}%
^{d}\left(  \frac{\partial\widetilde{a}_{kl}}{\partial\beta}\right)  ^{\top
}\frac{\partial\widetilde{a}_{kl}^{-1}}{\partial\beta}\right]  .
\]
By the stationary distribution, we get :

diagonal terms of $I^{\left(  1\right)  }$:%
\begin{align}
I_{k_{11}k_{11}}^{\left(  1\right)  }  &  =\frac{1}{2k_{11}\left(  1-\rho
^{2}\right)  }\label{fisher_info_alpha}\\
I_{k_{21}k_{21}}^{\left(  1\right)  }  &  =\frac{\sigma_{1}^{2}}%
{2k_{11}\left(  1-\rho^{2}\right)  \sigma_{2}^{2}}\nonumber\\
I_{k_{22}k_{22}}^{\left(  1\right)  }  &  =\frac{-2k_{11}k_{21}\rho\sigma
_{2}\sigma_{1}+k_{21}^{2}\sigma_{1}^{2}+k_{11}\left(  k_{11}+k_{22}\right)
\sigma_{2}^{2}}{2k_{11}k_{22}\left(  k_{11}+k_{22}\right)  \left(  1-\rho
^{2}\right)  \sigma_{2}^{2}}\nonumber\\
I_{\mu_{1}\mu_{1}}^{\left(  1\right)  }  &  =\frac{-2k_{11}k_{21}\rho
\sigma_{2}\sigma_{1}+k_{21}^{2}\sigma_{1}^{2}+k_{11}^{2}\sigma_{2}^{2}%
}{\left(  1-\rho^{2}\right)  \sigma_{1}^{2}\sigma_{2}^{2}}\nonumber\\
I_{\mu_{2}\mu_{2}}^{\left(  1\right)  }  &  =\frac{k_{22}^{2}}{\left(
1-\rho^{2}\right)  \sigma_{2}^{2}}\nonumber
\end{align}
off-diagonal terms of $I^{\left(  1\right)  }$:%
\begin{align*}
I_{k_{11}k_{21}}^{\left(  1\right)  }  &  =\frac{\rho\sigma_{1}}{2k_{11}%
\sigma_{2}\left(  \rho^{2}-1\right)  }\\
I_{k_{11}\mu_{1}}^{\left(  1\right)  }  &  =0\\
I_{k_{11}k_{22}}^{\left(  1\right)  }  &  =\frac{\rho\left(  k_{21}\sigma
_{1}-2k_{11}\rho\sigma_{2}\right)  }{2k_{11}\left(  k_{11}+k_{22}\right)
\left(  1-\rho^{2}\right)  \sigma_{2}}\\
I_{k_{21}\mu_{1}}^{\left(  1\right)  }  &  =0\\
I_{k_{22}\mu_{2}}^{\left(  1\right)  }  &  =0\\
I_{k_{21}k_{22}}^{\left(  1\right)  }  &  =\frac{\sigma_{1}\left(  2k_{11}%
\rho\sigma_{2}-k_{21}\sigma_{1}\right)  }{2k_{11}\left(  k_{11}+k_{22}\right)
\left(  1-\rho^{2}\right)  \sigma_{2}^{2}}\\
I_{k_{21}\mu_{2}}^{\left(  1\right)  }  &  =0\\
I_{k_{2}\mu_{1}}^{\left(  1\right)  }  &  =0\\
I_{\mu_{1}\mu_{2}}^{\left(  1\right)  }  &  =\frac{k_{22}\left(  k_{21}%
\sigma_{1}-k_{11}\rho\sigma_{2}\right)  }{\left(  1-\rho^{2}\right)
\sigma_{1}\sigma_{2}^{2}}\\
I_{k_{11}\mu_{2}}^{\left(  1\right)  }  &  =0
\end{align*}
diagonal terms of $I^{\left(  2\right)  }$:%
\begin{align}
I_{\sigma_{1}\sigma_{1}}^{\left(  2\right)  }  &  =\frac{2-\rho^{2}}{\left(
1-\rho^{2}\right)  \sigma_{1}^{2}}\label{fisher_info_beta}\\
I_{\sigma_{2}\sigma_{2}}^{\left(  2\right)  }  &  =\frac{2-\rho^{2}}{\left(
1-\rho^{2}\right)  \sigma_{2}^{2}}\nonumber\\
I_{\rho\rho}^{\left(  2\right)  }  &  =\frac{\rho^{2}+1}{\left(  1-\rho
^{2}\right)  ^{2}}\nonumber
\end{align}
off-diagonal terms of $I^{\left(  2\right)  }$:%
\begin{align*}
I_{\sigma_{1}\sigma_{2}}^{\left(  2\right)  }  &  =\frac{\rho^{2}}{\sigma
_{1}\sigma_{2}\left(  \rho^{2}-1\right)  }\\
I_{\sigma_{1}\rho}^{\left(  2\right)  }  &  =\frac{\rho}{\sigma_{1}\left(
\rho^{2}-1\right)  }\\
I_{\sigma_{2}\rho}^{\left(  2\right)  }  &  =\frac{\rho}{\sigma_{2}\left(
\rho^{2}-1\right)  }%
\end{align*}
the above results for $\sigma_{1},\sigma_{2},\rho$ is combined in%
\[
I(\sigma_{2},\rho,\sigma_{1})=\left(
\begin{array}
[c]{ccc}%
\frac{2-\rho^{2}}{\left(  1-\rho^{2}\right)  \sigma_{2}^{2}} & -\frac{\rho
}{\left(  1-\rho^{2}\right)  \sigma_{2}} & -\frac{\rho^{2}}{\left(  1-\rho
^{2}\right)  \sigma_{1}\sigma_{2}}\\
-\frac{\rho}{\left(  1-\rho^{2}\right)  \sigma_{2}} & \frac{1+\rho^{2}%
}{\left(  1-\rho^{2}\right)  ^{2}} & -\frac{\rho}{\left(  1-\rho^{2}\right)
\sigma_{1}}\\
-\frac{\rho^{2}}{\left(  1-\rho^{2}\right)  \sigma_{1}\sigma_{2}} &
-\frac{\rho}{\left(  1-\rho^{2}\right)  \sigma_{1}} & \frac{2-\rho^{2}%
}{\left(  1-\rho^{2}\right)  \sigma_{1}^{2}}%
\end{array}
\right)
\]


\subsubsection{$\mu$}

$I_{\mu_{1}\mu_{1}}^{\left(  1\right)  }=\frac{-2k_{11}k_{21}\rho\sigma
_{2}\sigma_{1}+k_{21}^{2}\sigma_{1}^{2}+k_{11}^{2}\sigma_{2}^{2}}{\left(
1-\rho^{2}\right)  \sigma_{1}^{2}\sigma_{2}^{2}}$ $I_{\mu_{2}\mu_{2}}^{\left(
1\right)  }=\frac{k_{22}^{2}}{\left(  1-\rho^{2}\right)  \sigma_{2}^{2}}$
$I_{\mu_{1}\mu_{2}}^{\left(  1\right)  }=\frac{k_{22}\left(  k_{21}\sigma
_{1}-k_{11}\rho\sigma_{2}\right)  }{\left(  1-\rho^{2}\right)  \sigma
_{1}\sigma_{2}^{2}}$, so%
\begin{align*}
&  (1,1)I_{\mu_{1},\mu_{2}}(1,1)^{\intercal}\\
&  =\frac{-2k_{11}k_{21}\rho\sigma_{2}\sigma_{1}+k_{21}^{2}\sigma_{1}%
^{2}+k_{11}^{2}\sigma_{2}^{2}}{\left(  1-\rho^{2}\right)  \sigma_{1}^{2}%
\sigma_{2}^{2}}+\frac{k_{22}^{2}}{\left(  1-\rho^{2}\right)  \sigma_{2}^{2}%
}+\frac{2k_{22}\left(  k_{21}\sigma_{1}-k_{11}\rho\sigma_{2}\right)  }{\left(
1-\rho^{2}\right)  \sigma_{1}\sigma_{2}^{2}}\\
&  =\frac{-2k_{11}k_{21}\rho\sigma_{2}\sigma_{1}+k_{21}^{2}\sigma_{1}%
^{2}+k_{11}^{2}\sigma_{2}^{2}+k_{22}^{2}\sigma_{1}^{2}+2k_{22}\left(
k_{21}\sigma_{1}^{2}-k_{11}\rho\sigma_{1}\sigma_{2}\right)  }{\left(
1-\rho^{2}\right)  \sigma_{1}^{2}\sigma_{2}^{2}}\\
&  =\frac{-2k_{11}(k_{21}+k_{22})\rho\sigma_{2}\sigma_{1}+k_{11}^{2}\sigma
_{2}^{2}+\left(  k_{21}^{2}+k_{22}^{2}+2k_{22}k_{21}\right)  \sigma_{1}^{2}%
}{\left(  1-\rho^{2}\right)  \sigma_{1}^{2}\sigma_{2}^{2}}%
\end{align*}
then $k_{11}$ as $\kappa_{2}$, $k_{22}=-k_{21}$ as $\kappa_{1}$, reverse the
order of $\sigma_{1}$ and $\sigma_{2}$ , we get%
\begin{align*}
&  (1,1)I_{\mu_{1},\mu_{2}}(1,1)^{\intercal}=\frac{-2k_{11}(k_{21}+k_{22}%
)\rho\sigma_{2}\sigma_{1}+k_{11}^{2}\sigma_{2}^{2}+\left(  k_{21}^{2}%
+k_{22}^{2}+2k_{22}k_{21}\right)  \sigma_{1}^{2}}{\left(  1-\rho^{2}\right)
\sigma_{1}^{2}\sigma_{2}^{2}}\\
&  =\frac{k_{11}^{2}\sigma_{2}^{2}}{\left(  1-\rho^{2}\right)  \sigma_{1}%
^{2}\sigma_{2}^{2}}\text{ and reverse the order of }\sigma_{1},\sigma_{2}\\
&  =\frac{\kappa_{2}^{2}}{\left(  1-\rho^{2}\right)  \sigma_{2}^{2}}%
\end{align*}


\subsubsection{ $\kappa_{1}$}%

\begin{align}
I_{k_{21}k_{21}}^{\left(  1\right)  }  &  =\frac{\sigma_{1}^{2}}%
{2k_{11}\left(  1-\rho^{2}\right)  \sigma_{2}^{2}}\nonumber\\
I_{k_{22}k_{22}}^{\left(  1\right)  }  &  =\frac{-2k_{11}k_{21}\rho\sigma
_{2}\sigma_{1}+k_{21}^{2}\sigma_{1}^{2}+k_{11}\left(  k_{11}+k_{22}\right)
\sigma_{2}^{2}}{2k_{11}k_{22}\left(  k_{11}+k_{22}\right)  \left(  1-\rho
^{2}\right)  \sigma_{2}^{2}}\nonumber
\end{align}
$I_{k_{21}k_{22}}^{\left(  1\right)  }=\frac{\sigma_{1}\left(  2k_{11}%
\rho\sigma_{2}-k_{21}\sigma_{1}\right)  }{2k_{11}\left(  k_{11}+k_{22}\right)
\left(  1-\rho^{2}\right)  \sigma_{2}^{2}}$%
\begin{align*}
&  (1,-1)I_{k_{22},k_{21}}(1,-1)^{\intercal}\\
&  =\frac{\sigma_{1}^{2}}{2k_{11}\left(  1-\rho^{2}\right)  \sigma_{2}^{2}%
}+\frac{-2k_{11}k_{21}\rho\sigma_{2}\sigma_{1}+k_{21}^{2}\sigma_{1}^{2}%
+k_{11}\left(  k_{11}+k_{22}\right)  \sigma_{2}^{2}}{2k_{11}k_{22}\left(
k_{11}+k_{22}\right)  \left(  1-\rho^{2}\right)  \sigma_{2}^{2}}-\frac
{2\sigma_{1}\left(  2k_{11}\rho\sigma_{2}-k_{21}\sigma_{1}\right)  }%
{2k_{11}\left(  k_{11}+k_{22}\right)  \left(  1-\rho^{2}\right)  \sigma
_{2}^{2}}%
\end{align*}
then $k_{11}$ as $\kappa_{2}$, $k_{22}=-k_{21}$ as $\kappa_{1}$,
\begin{align*}
&  (1,-1)I_{k_{22},k_{21}}(1,-1)^{\intercal}\\
&  =\frac{\sigma_{1}^{2}}{2\kappa_{2}\left(  1-\rho^{2}\right)  \sigma_{2}%
^{2}}+\frac{2\kappa_{1}\kappa_{2}\rho\sigma_{2}\sigma_{1}+\kappa_{1}^{2}%
\sigma_{1}^{2}+\kappa_{2}\left(  \kappa_{1}+\kappa_{2}\right)  \sigma_{2}^{2}%
}{2\kappa_{1}\kappa_{2}\left(  \kappa_{1}+\kappa_{2}\right)  \left(
1-\rho^{2}\right)  \sigma_{2}^{2}}-\frac{\sigma_{1}\left(  2\rho\kappa
_{2}\sigma_{2}+\kappa_{1}\sigma_{1}\right)  }{\kappa_{2}\left(  \kappa
_{1}+\kappa_{2}\right)  \left(  1-\rho^{2}\right)  \sigma_{2}^{2}}\\
&  =\frac{\sigma_{1}^{2}}{2\kappa_{2}\left(  1-\rho^{2}\right)  \sigma_{2}%
^{2}}+\frac{\kappa_{2}\rho\sigma_{2}\sigma_{1}+\frac{1}{2}\kappa_{1}\sigma
_{1}^{2}+\frac{1}{2}\kappa_{2}\left(  \kappa_{1}+\kappa_{2}\right)  \sigma
_{2}^{2}}{\kappa_{2}\left(  \kappa_{1}+\kappa_{2}\right)  \left(  1-\rho
^{2}\right)  \sigma_{2}^{2}}-\frac{2\rho\kappa_{2}\sigma_{1}\sigma_{2}%
+\kappa_{1}\sigma_{1}^{2}}{\kappa_{2}\left(  \kappa_{1}+\kappa_{2}\right)
\left(  1-\rho^{2}\right)  \sigma_{2}^{2}}%
\end{align*}
reverse the order of $\sigma_{1}$ and $\sigma_{2}$ , we get

\bigskip

..... to check later.....
\end{document}